%%%%\input{docs/tbl_genbank_summary_info_genomes}

% here we have an example of a latex formatted table
\begin{table}[!tch]
\centering
\begin{tabular}{lcllrrrrrrrr} \footnotesize % to get a small font and fit table within the margins...
 Species  & Type  & RefSeq ID      & INSDC       & Size (Mb) & GC\%    & Protein  & rRNA  & tRNA  & Other RNA & Gene   & Pseudogene \\ \hline\hline
 \Ecol    & Chr   & NC\_000913.3   & U00096.3    & 4.64      & 50.8    & 4,140    & 22    & 89    &  67       & 4,498  & 184        \\
 \Bsub    & Chr   & NC\_000964.3   & AL009126.3  & 4.22      & 43.5    & 4,174    & 30    & 86    &  62       & 4,420  &  68        \\
 \Mgen    & Chr   & NC\_000908.2   & L43967.2    & 0.58      & 31.7    &   515    &  3    & 36    &   3       &   566  &   9        \\
 \Mpne    & Chr   & NC\_000912.1   & U00089.2    & 0.82      & 40.0    &   691    &  3    & 37    & 315       & 1,061  &  15        \\ \hline
\end{tabular}

\caption[Genome sequence information for four bacteria species downloaded from \GB]{%
\label{tbl:genbankgenomesinfo}\textbf{Genome sequence information for four bacteria species downloaded from \GB.} Whole-genome summary table showing number of annotated features, such genes and proteins, along with sequence characteristics, such as size and average GC content. 
}%caption

\end{table}


% here we have an example of a latex formatted table
\begin{table}[!tch]
\centering
\begin{tabular}{lcllrrrrrrrr} \footnotesize % to get a small font and fit table within the margins...
 Species  & Type  & RefSeq ID      & INSDC       & Size (Mb) & GC\%    & Protein  & rRNA  & tRNA  & Other RNA & Gene   & Pseudogene \\ \hline\hline
 \Ecol    & Chr   & NC\_000913.3   & U00096.3    & 4.64      & 50.8    & 4,140    & 22    & 89    &  67       & 4,498  & 184        \\
 \Bsub    & Chr   & NC\_000964.3   & AL009126.3  & 4.22      & 43.5    & 4,174    & 30    & 86    &  62       & 4,420  &  68        \\
 \Mgen    & Chr   & NC\_000908.2   & L43967.2    & 0.58      & 31.7    &   515    &  3    & 36    &   3       &   566  &   9        \\
 \Mpne    & Chr   & NC\_000912.1   & U00089.2    & 0.82      & 40.0    &   691    &  3    & 37    & 315       & 1,061  &  15        \\ \hline
\end{tabular}

\caption[Genome sequence information for four bacteria species downloaded from \GB]{%
\label{tbl:genbankgenomesinfo}\textbf{Genome sequence information for four bacteria species downloaded from \GB.} Whole-genome summary table showing number of annotated features, such genes and proteins, along with sequence characteristics, such as size and average GC content. 
}%caption

\end{table}


% here we have an example of a latex formatted table
\begin{table}[!tch]
\centering
\begin{tabular}{lcllrrrrrrrr} \footnotesize % to get a small font and fit table within the margins...
 Species  & Type  & RefSeq ID      & INSDC       & Size (Mb) & GC\%    & Protein  & rRNA  & tRNA  & Other RNA & Gene   & Pseudogene \\ \hline\hline
 \Ecol    & Chr   & NC\_000913.3   & U00096.3    & 4.64      & 50.8    & 4,140    & 22    & 89    &  67       & 4,498  & 184        \\
 \Bsub    & Chr   & NC\_000964.3   & AL009126.3  & 4.22      & 43.5    & 4,174    & 30    & 86    &  62       & 4,420  &  68        \\
 \Mgen    & Chr   & NC\_000908.2   & L43967.2    & 0.58      & 31.7    &   515    &  3    & 36    &   3       &   566  &   9        \\
 \Mpne    & Chr   & NC\_000912.1   & U00089.2    & 0.82      & 40.0    &   691    &  3    & 37    & 315       & 1,061  &  15        \\ \hline
\end{tabular}

\caption[Genome sequence information for four bacteria species downloaded from \GB]{%
\label{tbl:genbankgenomesinfo}\textbf{Genome sequence information for four bacteria species downloaded from \GB.} Whole-genome summary table showing number of annotated features, such genes and proteins, along with sequence characteristics, such as size and average GC content. 
}%caption

\end{table}


% here we have an example of a latex formatted table
\begin{table}[!tch]
\centering
\begin{tabular}{lcllrrrrrrrr} \footnotesize % to get a small font and fit table within the margins...
 Species  & Type  & RefSeq ID      & INSDC       & Size (Mb) & GC\%    & Protein  & rRNA  & tRNA  & Other RNA & Gene   & Pseudogene \\ \hline\hline
 \Ecol    & Chr   & NC\_000913.3   & U00096.3    & 4.64      & 50.8    & 4,140    & 22    & 89    &  67       & 4,498  & 184        \\
 \Bsub    & Chr   & NC\_000964.3   & AL009126.3  & 4.22      & 43.5    & 4,174    & 30    & 86    &  62       & 4,420  &  68        \\
 \Mgen    & Chr   & NC\_000908.2   & L43967.2    & 0.58      & 31.7    &   515    &  3    & 36    &   3       &   566  &   9        \\
 \Mpne    & Chr   & NC\_000912.1   & U00089.2    & 0.82      & 40.0    &   691    &  3    & 37    & 315       & 1,061  &  15        \\ \hline
\end{tabular}

\caption[Genome sequence information for four bacteria species downloaded from \GB]{%
\label{tbl:genbankgenomesinfo}\textbf{Genome sequence information for four bacteria species downloaded from \GB.} Whole-genome summary table showing number of annotated features, such genes and proteins, along with sequence characteristics, such as size and average GC content. 
}%caption

\end{table}
